\chapter{The Extent and Military Force of the Empire in the Age  of the Antonines.}



\marginnote{Introdction}

\lettrine[lines=3]{I}n the ſecond century of the Chriſtian Æra, the empire of Rome comprehended the faireſt part of the earth, and the moſt civilized portion of mankind. The frontiers of that extenſive monarchy were guarded by ancient renown and diſciplined valour. The gentle, but powerful influence of laws and manners had gradually cemented the union of the provinces. Their peaceful inhabitants enjoyed and abuſed the advantages of wealth and luxury. The image of a free conſtitution was preſerved with decent reverence. The Roman ſenate appeared to poſſeſs the ſovereign authority, and devolved on the emperors all the executive powers of government. \marginnote{A. D. 98\textemdash180} During a happy period of more than fourſcore years, the public adminiſtration was conducted by the virtue and abilities of Trajan, Hadrian, and the two Antonines. It is the deſign of this and of the two ſucceeding chapters, to deſcribe the proſperous condition of their empire ; and afterwards, from the death of Marcus Antoninus, to deduce the moſt important circumſtances of its decline and fall; a revolution which will ever be remembered, and is ſtill felt by the nations of the earth

\marginnote{Moderation of Augſtus.} The principal conqueſts of the Romans were atchieved under the republic; and the emperors, for the moſt part, were ſatisſied with preſerving thoſe dominions which had been acquired by the policy of the ſenate, the active emulation of the conſuls, and the martial enthuſiaſm of the people. The ſeven firſt centuries were filled with a rapid ſucceſſion of triumphs; but it was reſerved for Auguſtus, to relinquiſh the ambitious deſign of ſubduing the whole earth, and to introduce a ſpirit of moderation into the public councils. Inclined to peace by his temper and ſituation, it was eaſy for him to diſcover, that Rome, in her preſent exalted ſituation, had much leſs to hope than to fear from the chance of arms; and that, in the proſecution of remote wars, the undertaking became every day more difficult, the event more doubtful, and the poſſeſſion more precarious, and leſs beneficial. The experience of Auguſtus added weight to theſe ſalutary reflections. Inſtead of expoſing his perſon and his legions to the arrows of the Parthians, he fatisfied himſelf with the reſtitution of the ſtandards and priſoners which were taken in the defeat of Craſſus \footnote{\lettrine[lines=3]{D}ION CASSIUS, (1. liv. p. 736.) with the annotations of Reymar, who has collected all that Roman vanity has left upon the ſubjeiſt. The marble of Ancyra, on which Auguſtus recorded his own exploits, aſſerts that \emph{he compelled} the Parthians to reſtore the enſigns of Craſſus.}.






