\chapter{The Extent and Military Force of the Empire in the Age  of the Antonines.}



\marginnote{Introdction}

\lettrine[lines=3]{I}n the second century of the Christian Æra, the empire of Rome comprehended the fairest part of the earth, and the most civilized portion of mankind. The frontiers of that extensive monarchy were guarded by ancient renown and disciplined valour. The gentle, but powerful influence of laws and manners had gradually cemented the union of the provinces. Their peaceful inhabitants enjoyed and abused the advantages of wealth and luxury. The image of a free constitution was preserved with decent reverence. The Roman senate appeared to possess the sovereign authority, and devolved on the emperors all the executive powers of government. \marginnote{A. D. 98\textemdash180} During a happy period of more than fourscore years, the public administration was conducted by the virtue and abilities of Trajan, Hadrian, and the two Antonines. It is the design of this and of the two succeeding chapters, to describe the prosperous condition of their empire ; and afterwards, from the death of Marcus Antoninus, to deduce the most important circumstances of its decline and fall; a revolution which will ever be remembered, and is still felt by the nations of the earth

\marginnote{Moderation of Augstus.} The principal conquests of the Romans were atchieved under the republic; and the emperors, for the most part, were satissied with preserving those dominions which had been acquired by the policy of the senate, the active emulation of the consuls, and the martial enthusiasm of the people. The seven first centuries were filled with a rapid succession of triumphs; but it was reserved for Augustus, to relinquish the ambitious design of subduing the whole earth, and to introduce a spirit of moderation into the public councils. Inclined to peace by his temper and situation, it was easy for him to discover, that Rome, in her present exalted situation, had much less to hope than to fear from the chance of arms; and that, in the prosecution of remote wars, the undertaking became every day more difficult, the event more doubtful, and the possession more precarious, and less beneficial. The experience of Augustus added weight to these salutary reflections. Instead of exposing his person and his legions to the arrows of the Parthians, he fatisfied himself with the restitution of the standards and prisoners which were taken in the defeat of Crassus \footnote{\lettrine[lines=3]{D}ION CASSIUS, (1. liv. p. 736.) with the annotations of Reymar, who has collected all that Roman vanity has left upon the subjeist. The marble of Ancyra, on which Augustus recorded his own exploits, asserts that \emph{he compelled} the Parthians to restore the ensigns of Crassus.}.


His generals, in the early part of his reign, attempted the reduction of Æthiopia and Arabia Felix. They marched near a thousand miles to the fouth of the tropic; but the heat of the climate soon repelled the invaders, and protected the unwarlike natives of those sequestered regions \footnote{ Strabo, (l. xvi. p. 780.) Pliny the elder, (Hist. Natur. l. vi. c. 32. 35.]) and Dion Cassius, (l. liii. p. 723, and l. liv. p. 734.) have left us very curious details concerning these wars. The Romans made themselves masters of Mariaba, or Merab, a city of Arabia Fælix, well known to the Orientals. (See Abulseda and the Nubian geography, p. 52). They were arrived within three days’ journey of the Spice country, the rich object of their invasion.} The northern countries of Europe scarcely deserved the expence and labour of the conquest. The forests and morasses of Germany were filled with a hardy race of barbarians, who despised life when it was separated from freedom; and though, on the first attack, they seemed to yield to the weight of the Roman power, they soon, by a signal act of despair, regained their independence, and reminded Augustus of the vicissitude of fortune\footnote{By the slaughter of Varus and his three legions. See the first book of the Annals of Tacitus. Sueton. in August. c. 23. and Velleius Paterculus, l. ii. c. 117, \&c. Augustus did not receive the melancholy news with all the temper and firmness that might have been expected from his character.}. On the death of that emperor, his testament was publickly read in the senate. He bequeathed, as a valuable legacy to his successors, the advice of confining the empire within those limits, which Nature seemed to have placed as its permanent bulwarks and boundaries; on the weft the Atlantic ocean; the Rhine and Danube on the north; the Euphrates on the east; and towards the south, the sandy deserts of Arabia and Africa\footnote{Tacit. Annal. l. ii. Dion Cassius, l. lvi. p. 833, and the speech of Augustus himself, in Julian’s Cæsars. It receives great light from the learned notes of his French translator, M. Spanheim.}.

\marginnote{Imitated by his successors.}
Happily for the repose of mankind, the moderate system recommended by the wisdom of Augustus, was adopted by the fears and vices of his immediate successors. Engaged in the pursuit of pleasure, or in the exercise of tyranny, the first Caesars seldom shewed themselves to the armies, or to the provinces; nor were they disposed to suffer, that those triumphs which their indolence neglected, should be usurped by the conduct and valour of their lieutenants. The military fame of a subject was considered as an insolent invasion of the Imperial prerogative; and it became the duty, as well as interest of every Roman general, to guard the frontiers entrusted to his care, without aspiring to conquests which might have proved no less fatal to himself than to the vanquished barbarians\footnote{Germanicus, Suetonius Paulinus, and Agricola were checked and recalled in the course of their victories. Corbulo was put to death. Military merit, as it is admirably expressed by Tacitus, was, in the strictest sense of the word, \emph{\foreignlanguage{latin}{imperatoria virtus}}.}.

\marginnote{Conquest of Britain was the first exception to it.}
The only accession which the Roman empire received, during the first century of the Christian Æra was the province of Britain.
In this single instance the successors of Cæsar and Augustus were persuaded to follow the example of the former, rather than the rather than the precept of the latter. The proximity of its situation to the coast of Gaul seemed to invite their arms; the pleasing, though doubtful intelligence of a pearl fishery, attracted their avarice\footnote{Cæsar himself conceals that ignoble motive; but it is mentioned by Suetonius, c. 47. The British pearls proved, however, of little value, on account of their dark and livid color. Tacitus observes, with reason, (in Agricola, c. 12.) that it was an inherent defect. \foreignlanguage{latin}{“Ego facilius crediderim, naturam margaritis deesse quam nobis avaritiam.”}}; and as Britain was viewed in the light of a distinct and insulated world, the conquest fearcely formed any exception to the general system of continental measures. After a war of about forty years, undertaken by the most stupid\footnote{Claudius, Nero, and Domitian. A hope is expressed by Pomponius Mela, l. iii. c. 6. (he wrote under Claudius) that, by the success of the Roman arms, the island and its savage inhabitants would soon be better known. It is amusing enough to peruse such passages in the midst of London.}, maintained by the most dissolute, and terminated by the most timid of all the emperors, the far greater part of the island submitted to the Roman yoke\footnote{See the admirable abridgment given by Tacitus, in the life of Agricola, and copiously, though perhaps not completely illustrated by our own antiquarians, Camden and Horsley.}. The various tribes of Britons possessed valour without conduct, and the love of freedom without the spirit of union. They took up arms with savage fierceness; they laid them down, or turned them against each other with wild inconstancy; and while they fought singly, they were successively subdued. Neither the fortitude of Caratacus, nor the despair of Boadicea, nor the fanaticism of the Druids could avert the slavery of their country, or resist the steady progress of the Imperial generals, who maintained the national glory, when the throne was disgraced by the weakest, or the most vicious of mankind. At the very time when Domitian, confined to his palace, felt the terrors which he inspired; his legions, under the command of the virtuous Agricola, defeated the collected force of the Caledonians, at the foot of the Grampian hills; and his fleets, venturing to explore an unknown and dangerous navigation, displayed the Roman arms round every part of the island. The conqueft of Britain was considered as already atchieved; and it was the design of Agricola to complete and ensure his success, by the easy reduction of Ireland, for which, in his opinion, one legion and a few auxiliaries were sufficient\footnote{The Irish writers, jealous of their national honor, are extremely provoked on this occasion, both with Tacitus and with Agricola.}. The western isle might be improved into a valuable possession, and the Britons would wear their chains with the less reluctance, if the prospect and example of freedom was on every side removed from before their eyes.

But the superior merit of Agricola soon occasioned his removal from the government of Britain; and for ever disappointed this rational, though extensive scheme of conquest. Before his departure, the prudent general had provided for security as well as for dominion. He had observed, that the island is almost divided into two unequal parts, by the opposite gulfs, or as they are now called, the Firths of Scotland. Across the narrow interval of about forty miles, he had drawn a line of military stations, which was afterwards fortified in the reign of Antoninus Pius, by a turf rampart erected on foundations of stone\footnote{See Horsley’s Britannia Romana, l.i.c. 10.}. This wall of Antoninus, at a small distance beyond the modern cities of Edinburgh and Glasgow, was fixed as the limit of the Roman province. The native Caledonians preserved in the northern extremity of the island their wild independence, for which they were not less indebted to their poverty than to their valour. Their incursions were frequently repelled and chastised; but their country was never subdued\footnote{The poet Buchanan celebrates, with elegance and spirit (see his Sylvæ
  v.) the unviolated independence of his native country. But, if the single testimony of Richard of Cirencester was sufficient to create a Roman province of Vespasiana to the north of the wall, that independence would be reduced within very narrow limits.}. The matters of the fairest and most wealthy climates of the globe, turned with contempt from gloomy hills assailed by the winter tempest, from lakes concealed in a blue mist, and from cold and lonely heaths, over which the deer of the forest were chased by a troop of naked barbarians\footnote{See Appian (in Proæm) and the uniform imagery of Ossaian’s poems, which according to every hypothesis, were composed by a native Caledonian.}.

\marginnote{Conquest of Dacia; the second exception.}
Such was the state of the Roman frontiers, and such the maxims of imperial policy from the death of Augustus to the accession of Trajan. That virtuous and active prince had received the education of a soldier, and possessed the talents of a general\footnote{See Pliny’s Panegyric, which seems founded on facts.}. The peaceful system of his predecessors was interrupted by scenes of war and conquest; and the legions, after a long interval, beheld a military emperor at their head. The first exploits of Trajan were against the Dacians, the most warlike of men, who dwelt beyond the Danube, and who, during the reign of Domitian, had insulted with impunity the Majefty of Rome\footnote{Dion Cassius, l. lxvii.}. To the strength and fierceness of barbarians, they added a contempt for life, which was derived from a warm persuasion of the immortality and transmigration of the soul\footnote{Herodotus, l. iv. c. 94. Julian in the Cæsars, with Spanheim’s observations.}. Decebalus, the Dacian king, approved himself a rival not unworthy of Trajan; nor did he despair of his own and the public fortune, till, by the confession of his enemies, he had exhausted every resource both of valour and policy\footnote{Plin. Epist. viii. 9.}. This memorable war, with a very short suspension of hostilities, lasted five years; and as the emperor could exert, without controul, the whole force of the state, it was terminated by an absolute submission of the barbarians\footnote{Dion Cassius, l. lxviii. p. 1123. 1131. Julian in Cæsaribus. Eutropius, viii. 2. 6. Aurelius Victor in Epitome.}. The new province of Dacia, which formed a second exception to the precept of Augustus, was about thirteen hundred  miles in circumference. Its natural boundaries were the Niester, the Teyss, or Tibiscus, the Lower Danube, and the Euxine Sea. The vestiges of a military road may still be traced from the banks of the Danube to the neighbourhood of Bender, a place famous in modern history, and the actual frontier of the Turkish and Russian empires\footnote{See a Memoir of M. Danville, on the Province of Dacia, in the Academie des Inscriptions, tom. xxviii. p. 444\textemdash468.}.

\marginnote{Conquests of Trajan in the east.}


